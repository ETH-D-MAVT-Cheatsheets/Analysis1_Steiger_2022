% !TeX root = ../../ZF_bmicha_Ana.tex
\subsection{Folgen und Reihen}
    \begin{description}
        \item[konvergent] Es existiert ein Grenzwert sonst divergent.
        \item[beschränkt] Alle Glieder in \emph{endlich breitem}, \emph{waagerechten Parallelstreifen} enthalten.
        \item[monoton wachsend] $\quad a_{n+1} \geq a_n \qquad  (\emph{strikt}:\,\, >)$
    \end{description}
    \fbox{mon. wachsend/ fallend \& beschränkt $\Rightarrow$ konvergent}
    \subsubsection{Rechenregeln für konvergente Folgen}
        Falls $\displaystyle \lim_{n\to \infty} a_n = a$ und $\displaystyle \lim_{n\to \infty} b_n = b$, gilt:
        \begin{align*}
            \lim_{n \to \infty}(a_n \pm b_n) &= a \pm b\\
            \lim_{n \to \infty}(a_n \cdot b_n) &= a \cdot b\\
            \lim_{n \to \infty}(a_n / b_n) &= a / b.
        \end{align*}
        Gilt auch für Funktionen; sofern Grenzwert existiert.
        % \vspace*{-1em}
    \subsubsection{Geometrische Reihe}
        \vspace{-1.5em}
        \begin{align*}
            \sum_{n=0}^{k}a \cdot q^n &= a \cdot \frac{1 - q^{k+1}}{1-q\phantom{^{k+1}}}\\
            \sum_{n=0}^{\infty}a \cdot q^n &= \frac{a}{1-q}, \quad \textrm{falls } \abs*{q} < 1
        \end{align*}
        \vspace*{-0.5em}
    \subsubsection{Arithmetische Reihe}
        \vspace{-1em}
        $$
            \sum_{n=1}^{\infty} \ [a_1 + (n-1) \cdot d] = \frac{n}{2} \cdot (a_1 + a_n)
        $$
    \subsubsection{Good to Know}
        \vspace{0em}
        \begin{itemize}
            \item $\displaystyle \sum_{n=1}^{k} n = \frac{k \cdot (k + 1)}{2} $ \hspace*{4em} $\triangleright \displaystyle \sum_{n=0}^{\infty} \frac{1}{n!} = e$
            % \item $\displaystyle \sum_{n=0}^{\infty} \frac{1}{n!} = e$
            \item $\displaystyle \sum_{n=1}^{\infty} \frac{1}{n} = \infty$, (harm. Reihe)
        \end{itemize}
        
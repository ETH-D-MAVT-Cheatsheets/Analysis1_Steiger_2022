% !TeX root = ../../ZF_bmicha_Ana.tex
\subsection{Lineare DGL 1. Ordnung}
    \vspace{0.5em}
    \mathbox%[$\quad q(x) \vcentcolon=$ Störfunktion]
    {
        y' + p(x) \cdot y = q(x)
    }
    $\triangleright$ $q(x) \vcentcolon=$ Störfunktion
    
    \subsubsection{Homogen $\hfill q(x) \equiv 0$}
        \vspace{0.5em}
        \mathbox{
            y' + p(x) \cdot y = 0
        }
        $\triangleright$ Immer separierbar. (\ref{sec:separierbareDGL})
    \subsubsection{Inhomogen $\hfill q(x) \neq 0$}\label{sec:1.Ord-homogen}
        \vspace{0.5em}
        \mathbox{
            y' + p(x) \cdot y = q(x)
        }
        \begin{enumerate}
            \item $q(x) = 0$ setzen, homogene Lösung $y_h$ finden
            % $y_h' + p(x) \cdot y_h = 0$
            \item Partikuläre Lösung $y_p$ finden:
            \begin{itemize}
                \item Ansatz von Tabelle %(\ref{sec:linh-1orderTabelle})
                \item Lagrange-Methode %(\ref{sec:linh-1orderLagrange})
            \end{itemize}
        \end{enumerate}

        \subsubsubsection{Ansatz}
            \begin{enumerate}
                \item Ansatz für $y_p$ in inhomogene DGL einsetzen
                \item Koeffizientenvergleich
                \item $y = y_h + y_p$
            \end{enumerate}
            \begin{center}
                \renewcommand{\arraystretch}{1.25}
                \begin{tabular}{ll}
                    \hline
                    Störfunktion $q(x)$                                                                                                                       & Ansatz für $y_p(x)$                          \\ \hline\hline
                    Konstante                                                                                                                                 & $y_p = A$                                    \\
                    lineare Fkt.                                                                                                                              & $y_p = Ax + B$                               \\
                    Polynom Grad $n$                                                                                                                          & $y_p = Ax^n + Bx^{n-1} + \cdots$             \\
                    \begin{tabular}[c]{@{}l@{}}$A \sin(\omega x), A \cos( \omega x),$ \phantom{dirtl} \\ $A \sin(\omega x) + B \cos( \omega x)$\end{tabular} & $y_p = C \sin(\omega x) + D \cos( \omega x)$  \\
                    $A e^{{\color{blue}b}x}$                                                                                                                  & $y_p = B e^{{\color{blue}b}x}$               \\\hline
                \end{tabular}
                
                \vspace{0.5em}
                \textit{Ansatz funktioniert nicht $\to$ mit $x$ multiplizieren.}
            \end{center}

        \subsubsubsection{Lagrange-Methode}
            \begin{enumerate}
                \item Lagrange-Ansatz $y_p$:\ $y_h = C \cdots \to y_p = C(x) \cdots$
                \item Ansatz in inhomogene DGL einsetzen
                \item Nach $C'(x)$ auflösen und integrieren
                \item $C(x)$ in $y_p$ einsetzen
                \item $y = y_p$
            \end{enumerate}
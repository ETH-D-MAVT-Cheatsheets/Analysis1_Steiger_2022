% !TeX root = ../../ZF_bmicha_Ana.tex
\subsection{Flächenparametrisierungen}
Gängige Tricks falls die Fläche gegeben ist als:
    \begin{enumerate}
        \item $z = f(x,y)$
            $$
                \vec{r}(x,y) = \begin{pmatrix}
                    x\\y\\f(x,y)
                \end{pmatrix}
            $$
        \item $z = f(r,\varphi)$
            $$
                \vec{r}(r,\varphi) = \begin{pmatrix}
                    r\cos(\varphi)\\r\sin(\varphi)\\f(r,\varphi)
                \end{pmatrix}
            $$
        \item Rotationssymmetrische Fläche
            $$
                \vec{r}(t,\varphi) = \begin{bmatrix}
                    \mathrm{Rotations-}\\\mathrm{matrix}\\(\varphi)
                \end{bmatrix}
                \cdot
                \begin{pmatrix}
                    \mathrm{Param.}\\
                    \mathrm{Kurve}\\
                    \mathrm{(t)}
                \end{pmatrix}.
            $$
    \end{enumerate}
    \subsubsection{Rotationsmatrizen}
        \vspace{-1em}
        $$
        s_\varphi \vcentcolon= \sin(\varphi) \qquad c_\varphi \vcentcolon= \cos(\varphi)
        $$
        $$
        \begin{bmatrix}
            1 & 0 & 0\\
            0 & c_\varphi &-s_\varphi\\
            0 & s_\varphi & c_\varphi
        \end{bmatrix}_x
        \quad
        \begin{bmatrix}
            c_\varphi &0 & s_\varphi\\
            0 & 1 & 0\\
            -s_\varphi & 0 & c_\varphi
        \end{bmatrix}_y
        \quad\!
        \begin{bmatrix}
            c_\varphi &-s_\varphi& 0\\
            s_\varphi & c_\varphi& 0\\
            0 & 0 & 1
        \end{bmatrix}_z
        $$
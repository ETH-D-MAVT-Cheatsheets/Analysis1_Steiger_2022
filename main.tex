\documentclass[8pt]{mpscheatsheet}
\usepackage{amsmath, amssymb, mathtools, empheq, physics}
\usepackage{xcolor}
\usepackage{graphicx}
% for wild tikz drawings
\usepackage{tikz}
\usepackage{mathdots}
\usepackage{yhmath}
\usepackage{cancel}
\usepackage{color}
\usepackage{siunitx}
\usepackage{array}
\usepackage{multirow}
\usepackage{amssymb}
\usepackage{gensymb}
\usepackage{tabularx}
\usepackage{extarrows}
\usepackage{booktabs}
\usetikzlibrary{fadings}
\usetikzlibrary{patterns}
\usetikzlibrary{shadows.blur}
\usetikzlibrary{shapes}

\author{Micha Bosshart - bmicha@ethz.ch}
\title{Analysis I}

\input{setup/setup.tex}

% Remove Section Enumeration
% \setcounter{secnumdepth}{0}

\begin{document}
    \section{Funktionen}
        \input{src/Funktionen/folgen-reihen.tex}
        \subsection{Grenzwerte}
        % !TeX root = ../../ZF_bmicha_Ana.tex
\subsubsection{Bernoulli de L'Hopital}
    Falls $\displaystyle \lim_{x \to a} f(x) = \lim_{x \to a} g(x) = 0$ (oder $\pm \infty$), so gilt
    $$
         \lim_{x \to a} \frac{f(x)}{g(x)} = \lim_{x \to a} \frac{f'(x)}{g'(x)}.
    $$
        \input{src/Funktionen/landau_symbole.tex}
        \input{src/Funktionen/eigenschaften.tex}
        \input{src/Funktionen/asymptoten.tex}
        \input{src/Funktionen/hyperbolische_funktionen.tex}
    \section{Komplexe Zahlen}
        \input{src/Komplexe_Zahlen/basics.tex}
        \input{src/Komplexe_Zahlen/nullstellen-reeller-Polynome.tex}
    \section{Differentialrechnung}
        \input{src/Differentialrechnung/inverse_ableitung.tex}
        \input{src/Differentialrechnung/tangenten.tex}
        % !TeX root = ../../ZF_bmicha_Ana.tex
\subsection{1D Fehlerrechnung}
    Die berechnete Grösse $f$ ist abhängig von der gemessenen Grösse $x$.
    Die gemessene Grösse weicht mit dem Messfehler $\Delta x$ von der Realität ab.
    \begin{itemize}
        \item \textbf{Absoluter Fehler}
            \vspace*{-0.5em}
            \mathbox{
                \Delta f = f(x + \Delta x) -\! f(x) \quad \overset{\Delta x \to 0}{\longrightarrow} \quad \Delta f \approx f'(x)\ \Delta x
            }
        \item \textbf{Relativer Fehler}
            \vspace*{-0.5em}
            \mathbox{
                \frac{\Delta f}{f}
            }
    \end{itemize}
    \subsubsubsection{Bemerkungen}
        \vspace{0.5em}
        \begin{minipage}{0.54\linewidth}
            \centering \vspace{4pt}
            $1\%$ Genauigkeit
            $$
                \frac{\Delta x}{x} = 1\% = \frac{1}{100}
            $$          
        \end{minipage}
        \begin{minipage}{0.45\linewidth}
            \centering
            Messfehler von $1^\circ$
            $$
                \Delta \alpha = \frac{\pi}{180}
            $$
        \end{minipage}
        \input{src/Differentialrechnung/evolute.tex}
        \input{src/Differentialrechnung/kruemmung.tex}
        \vfill \null \columnbreak
        \section{Parametrisierungen}
        % !TeX root = ../../ZF_bmicha_Ana.tex
\subsection{Kreis / Ellipse}
    Ellipse mit Mittelpunkt $(x_o, y_o)$ und Halbachsen $a$ \& $b$.\\
    \underline{\textit{implizit}:}
    $$
        \left(
            \frac{x-x_o}{a}
        \right)^2
        +
        \left(
            \frac{y-y_o}{b}
        \right)^2
        = 1
    $$
    \underline{\textit{parametrisiert}:}
    \begin{align*}
        x(t) &= x_o + a \cdot \cos(t)\\
        y(t) &= y_o + b \cdot \sin(t)
    \end{align*}
    \section{Integralrechnung}
        % \input{src/Integralrechnung/leibnitz.tex}
        \input{src/Integralrechnung/hauptsatz_infinitesimalrechnung.tex}
        \input{src/Integralrechnung/partialbruchzerlegung.tex}
        \input{src/Integralrechnung/partielle-integration.tex}
        \vfill \null \columnbreak
        \input{src/Integralrechnung/bogenlaenge.tex}
        \input{src/Integralrechnung/flaechenberechnungen.tex}
        \input{src/Integralrechnung/rotationsvolumen.tex}
        \vfill \null \columnbreak
        \input{src/Integralrechnung/rotationsoberflaechen.tex}
        % !TeX root = ../../ZF_bmicha_Ana.tex
\subsection{Schwerpunkt / Trägheitsmoment}
    Sei $H(x)$ die Höhe des Fläche a.d.S. $x$.\\
    Sei $\sigma$ die Flächendichte $[kg/m^2]$.
    \begin{align*}
        \textrm{Fläche: }  A = \int_{x_1}^{x_2} H(x)\ dx\\
        \textrm{Masse: }  M = \int_{x_1}^{x_2} \sigma \cdot H(x)\ dx\\
        \textrm{Schwerpunkt: }  x_s = \frac{1}{M} \int_{x_1}^{x_2} x \cdot \sigma \cdot H(x)\ dx\\
        \textrm{Trägheitsmoment: }  I_y = \int_{x_1}^{x_2} x^2 \cdot \sigma \cdot H(x)\ dx
    \end{align*}
    \subsubsection{Trägheitsmoment}
        \vspace*{0.5em}
        \mathbox{
            \int (\textrm{Abstand zur Rotationsachse})^2 \cdot (\textrm{Masse})
        }
        \input{src/Integralrechnung/uneigentliche-integrale.tex}
    \section{Potenzreihen}
        % !TeX root = ../../ZF_bmicha_Ana.tex
Potenzreihe der Funktion $f(x)$ um den Punkt $x_o$:
\mathbox{
    f(x) = \sum_{n=0}^{\infty} a_n \cdot (x-x_o)^n
}
\begin{itemize}
    \item Höchstens eine Potenzreihe von $f$ um $x_o$ existiert.
    \item Konvergiert für $\abs*{x-x_o} < r$
\end{itemize}
        \input{src/Potenzreihen/konvergenzradius.tex}
        % !TeX root = ../../ZF_bmicha_Ana.tex
\subsection{Taylorreihen}
    Taylorentwicklung von $f(x)$ um $x_o$:
    \mathbox{
        f(x) = \sum_{n=0}^\infty \frac{f^{(n)}(x_o)}{n!} (x-x_o)^n
    }
    \begin{itemize}
    \item ungerade Fkt $\Leftrightarrow$ ungerade Indizes: $a_1x + a_3x^3 + \dots$
    \item gerade Fkt $\Leftrightarrow$ gerade Indizes: $a_0 + a_2x^2 + \dots$
    \end{itemize}

\subsubsection{Ableitung Un-/Gerader Funktionen}
    \begin{itemize}
        \item $g(x)$ sei \textbf{gerade}
            $$
                g'(0)=g^{(3)}(0)=g^{(5)}(0)=...=0
            $$
        \item $f(x)$ sei \textbf{ungerade}
            $$
                f''(0)=f^{(4)}(0)=f^{(6)}(0)=...=0
            $$
    \end{itemize}
    \cbreak
    
    % \section{Mehrdimensionale Fkt. - Diff. Rechnung}
    %     % !TeX root = ../../ZF_bmicha_Ana.tex
\subsection{Allgemeine Fehlerrechnung}
    Die berechnete Grösse $f$ ist abhängig von den gemessenen Grössen $x,y$.
    Die gemessenen Grössen weichen mit den Messfehlern $dx,dy$ von der Realität ab.
    \begin{itemize}
        \item \textbf{Totales Differential / Absoluter Fehler}
            \mathbox{
                df \approx f_x\ dx + f_y\ dy
            }
        \item \textbf{Relativer Fehler}
            \mathbox{
                \frac{df}{f}
            }
    \end{itemize}
    %     \input{src/Mehrdimensionale-Funktionen_Differentialrechnung/niveaulinien-flaechen.tex}
    %     % !TeX root = ../../ZF_bmicha_Ana.tex
\subsection{Gradient}
    \vspace{-0.75em}
    $$
        \grad(f(x,y,z)) = 
        \begin{pmatrix}
            f_x \\ f_y \\ f_z
        \end{pmatrix}
        =
        \begin{pmatrix}
            \frac{\partial}{\partial x} \\[0.4em] \frac{\partial}{\partial y} \\[0.4em] \frac{\partial}{\partial z}
        \end{pmatrix}
    $$
    \begin{itemize}
        \item Steht senkrecht auf Niveauflächen/ -linien.
        \item Zeigt in Richtung des grössten Anstiegs der Funktionswerte.
    \end{itemize}
    \vspace*{0.5em}
    \subsubsection{2D - $f(x,y)$}
        \includegraphics[width=\linewidth]{src/Mehrdimensionale-Funktionen_Differentialrechnung/grad_2D.pdf}
    \subsubsection{3D - $f(x,y,z)$}
        \begin{center}
            \includegraphics[width=0.8\linewidth]{src/Mehrdimensionale-Funktionen_Differentialrechnung/grad_3D.pdf}
        \end{center}
    %     \input{src/Mehrdimensionale-Funktionen_Differentialrechnung/richtungsableitung.tex}
    %     \input{src/Mehrdimensionale-Funktionen_Differentialrechnung/tangentialebenen.tex}
    %     \input{src/Mehrdimensionale-Funktionen_Differentialrechnung/extremalstellen-2D.tex}
    %     \subsection{Satz von Schwarz}
    \textit{Reihenfolge von partiellen Ableitungen einer stetigen Funktion spielt keine Rolle:  $f_{xxyyzz} = f_{xzyxyz}$}
    % $$
    %     f_{xxyyzz} = f_{xzyxyz}
    % $$
    % \vspace{-1em}
    % % \cbreak
    %     \input{src/Mehrdimensionale-Funktionen_Differentialrechnung/integrabilitaetsbedingung.tex}
    %     \input{src/Mehrdimensionale-Funktionen_Differentialrechnung/verallg_kettenregel.tex}
    % \section{Mehrdimensionale Fkt. - Int. Rechnung}
    %     \input{src/Mehrdimensionale-Funktionen_Integralrechnung/doppel-und-dreifachintegrale.tex}
    %     \input{src/Mehrdimensionale-Funktionen_Integralrechnung/dimensionsvergleich.tex}
    %     \input{src/Mehrdimensionale-Funktionen_Integralrechnung/koordinatentransformationen.tex}
    %     \input{src/Mehrdimensionale-Funktionen_Integralrechnung/ellipsenkoordinaten.tex}
    %     \input{src/Mehrdimensionale-Funktionen_Integralrechnung/jacobi.tex}
    %     \input{src/Mehrdimensionale-Funktionen_Integralrechnung/schwerpunkte.tex}
    %     \input{src/Mehrdimensionale-Funktionen_Integralrechnung/traegheitsmoment.tex}
    %     \input{src/Mehrdimensionale-Funktionen_Integralrechnung/leibnitz.tex}
    %     \input{src/Mehrdimensionale-Funktionen_Integralrechnung/oberflaenintegrale.tex}
    % % \cbreak
    %     \input{src/Mehrdimensionale-Funktionen_Integralrechnung/parametrisierungen-flaechen.tex}
    % \vfill \null \columnbreak
    % \section{Vektoranalysis}
    %     % !TeX root = ../../ZF_bmicha_Ana.tex
Potenzreihe der Funktion $f(x)$ um den Punkt $x_o$:
\mathbox{
    f(x) = \sum_{n=0}^{\infty} a_n \cdot (x-x_o)^n
}
\begin{itemize}
    \item Höchstens eine Potenzreihe von $f$ um $x_o$ existiert.
    \item Konvergiert für $\abs*{x-x_o} < r$
\end{itemize}
    %     \input{src/Vektoranalysis/divrot.tex}
    %     \vspace{-0.25em}
    %     \input{src/Vektoranalysis/fluss.tex}
    %     \input{src/Vektoranalysis/arbeit.tex}
    %     \input{src/Vektoranalysis/DGLs-und-vektorfelder.tex}
    % \section{Differentialgleichungen (DGL)}
    %     % !TeX root = ../../ZF_bmicha_Ana.tex
Potenzreihe der Funktion $f(x)$ um den Punkt $x_o$:
\mathbox{
    f(x) = \sum_{n=0}^{\infty} a_n \cdot (x-x_o)^n
}
\begin{itemize}
    \item Höchstens eine Potenzreihe von $f$ um $x_o$ existiert.
    \item Konvergiert für $\abs*{x-x_o} < r$
\end{itemize}
    %     \input{src/Differentialgleichungen/separierbare-DGL.tex}
    %     \input{src/Differentialgleichungen/substitutionen.tex}
    %     \input{src/Differentialgleichungen/orthogonaltrajektorien.tex}
    %     \vfill \null \columnbreak
    %     \input{src/Differentialgleichungen/enveloppe.tex}
    %     \input{src/Differentialgleichungen/linear-1-ordnung.tex}
    %     \input{src/Differentialgleichungen/spezielle-DGL.tex}
    %     \input{src/Differentialgleichungen/DGL-konst-Koeff.tex}
    %     \cbreak
    %     \input{src/Differentialgleichungen/Lagrange-2-Ordnung.tex}
    %     \input{src/Differentialgleichungen/Euler-DGL.tex}
    %     \cbreak
    %     \input{src/Differentialgleichungen/DGL-Systeme.tex}
    % \vfill \null \pagebreak
    
    \section{Appendix}
    \input{src/Appendix/nullstellen_reeller_polynome.tex}
    \input{src/Appendix/intcosnsinn.tex}
   
\end{document}
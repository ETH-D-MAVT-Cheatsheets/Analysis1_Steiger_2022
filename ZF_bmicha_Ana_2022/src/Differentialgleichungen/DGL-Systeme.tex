% !TeX root = ../../ZF_bmicha_Ana.tex
\subsection{DGL Systeme}
    \vspace{0.5em}
    \begin{minipage}{0.5\linewidth}
        \centering
        \textbf{Allgemein:}
        \begin{align*}
            y_1'(x) &= f_1(x,y_1,y_2)\\
            y_2'(x) &= f_2(x,y_1,y_2)
        \end{align*}
    \end{minipage}
    \begin{minipage}{0.49\linewidth}
        \centering
        \textbf{Autonom:}
        \begin{align*}
            y_1'(x) &= f_1(y_1,y_2)\\
            y_2'(x) &= f_2(y_1,y_2)
        \end{align*}
    \end{minipage}
    \vspace{0.5em}
    \begin{itemize}
        \item Ein DGL System heisst \textbf{autonom}, falls die zu den gesuchten Funktionen ($y_1, y_2$) gehörige Variable ($x$) nicht explizit vorkommt.
    \end{itemize}

    \subsubsection{Existenz- \& Eindeutigkeitssatz}
        Die Funktionen $f_1,f_2$ seien \textbf{stetig in $\boldsymbol{x}$} und \textbf{stetig partiell nach $\boldsymbol{y_1,y_2}$ differenzierbar}.
        \begin{itemize}
            \item[$\Rightarrow$] Zu vorgegebenem $x_o, y_{1,o}, y_{2,o}$ gibt es \textit{genau ein Paar} $f_1,f_2$, welches das System löst und das AWP $y_1(x_o) = y_{1,o}$ und $y_2(x_o) = y_{2,o}$ erfüllt.
        \end{itemize}
    \subsubsection{Lineare Autonome DGL Systeme mit konstanten Koeffizienten}
        \vspace{-0.5em}
        $$
            \dot{x} = A \cdot x + b
        $$
        \vspace{0em}
        $$
            \underbrace{\begin{pmatrix}
                \dot{x}_1\\ \dot{x}_2
            \end{pmatrix}}_{\dot{x}}
            =
            \underbrace{\begin{bmatrix}
                a_{11} & a_{12}\\ a_{21} & a_{22}
            \end{bmatrix}}_{A}
            \cdot
            \underbrace{\begin{pmatrix}
                x_1\\ x_2
            \end{pmatrix}}_{x}
            +
            \underbrace{\begin{pmatrix}
                b_1\\ b_2
            \end{pmatrix}}_{b}
        $$
    \begin{itemize}
        \item \textbf{Eliminationsverfahren}
            \begin{enumerate}
                \item Nach einer gesuchten Funktion auflösen, ableiten und einsetzen
                \item DGL 2. Ordnung lösen
                \item Zweite gesuchte Funktion bestimmen
            \end{enumerate}
        \item \textbf{Linalg-Methode}
            \begin{enumerate}
                \item Eigenwerte $\lambda_i$ (EW) bestimmen
                    $$
                        0 \overset{!}{=} \det (A - \lambda \mathbb{I})
                    $$
                \item Eigenvektoren $\vvec_i$ zu EW $\lambda_i$ bestimmen
                    $$
                        (A - \lambda_i \mathbb{I}) \cdot \vvec_i \overset{!}{=} 0
                    $$
                \item Lösung Konstruieren:
                \begin{itemize}
                    \item $\lambda_{1,2}$ reell
                        \mathbox{
                            \begin{pmatrix}
                                x\\y
                            \end{pmatrix}
                            = 
                            C_1 \cdot \vvec_1 \cdot e^{\lambda_1 t}
                            +
                            C_2 \cdot \vvec_2 \cdot e^{\lambda_2 t}
                        }
                    \item $\lambda_{1,2} = a \pm ib$ 
                        $$
                            \lambda_{1} = a + ib \quad \to \quad \vvec_1 =
                            \begin{pmatrix}
                                c\\id
                            \end{pmatrix}
                        $$
                        $$
                            w = 
                            \vvec_1 \cdot e^{\lambda_1 t}
                            =
                            \begin{pmatrix}
                                c\\id
                            \end{pmatrix}
                            e^{(a+ib)t}
                        $$
                        \mathbox{
                            \begin{pmatrix}
                                x\\y
                            \end{pmatrix}
                            = 
                            C_1 \cdot \textrm{Re}(\wvec)
                            +
                            C_2 \cdot \textrm{Im}(\wvec)
                        }

                \end{itemize}

            \end{enumerate}
            

    \end{itemize}
    
    \subsubsection{Gleichgewichtspunkte (GGWP)}
        $$
            \begin{pmatrix}
                \dot{x}\\ \dot{y}
            \end{pmatrix}
            =
            \begin{pmatrix}
                f_1(x,y)\\ f_2(x,y)
            \end{pmatrix}
            \overset{!}{=} 
            \begin{pmatrix}
                0\\0
            \end{pmatrix}
        $$
    \subsubsection{Stabilitätsverhalten}
        \vspace{-1em}
        \begin{align*}
            \dot{x} = f_1(x,y)\\
            \dot{y} = f_2(x,y)
        \end{align*}
        \vspace{-0.75em}
        \begin{enumerate}
            \item Falls nötig um GGWP $(x_o,y_o)$ linearisieren:
                  $$
                    \begin{pmatrix}
                        \dot{\xi}\\[0.5em] \dot{\eta}
                    \end{pmatrix}
                    =
                    \underbrace{\begin{bmatrix}
                        \frac{\partial f_1}{\partial x}(x_o, y_o) & \frac{\partial f_1}{\partial y}(x_o, y_o)\\[0.5em]
                        \frac{\partial f_2}{\partial x}(x_o, y_o) & \frac{\partial f_2}{\partial y}(x_o, y_o)
                    \end{bmatrix}}_{A}
                    \cdot 
                    \begin{pmatrix}
                        \xi\\[0.5em] \eta
                    \end{pmatrix}
                  $$
            \item Eigenwerte $\lambda_i$ von $A$ bestimmen
            \begin{itemize}
                \item Re$(\lambda_i) < 0, \ \forall i \quad \to $ \textit{asymptotisch stabil}
                \item Re$(\lambda_i) \leq 0, \ \forall i \quad \to $ \textit{stabil}
                \item Re$(\lambda_i) > 0, \ \exists i \quad \to $ \textit{instabil}
            \end{itemize}
        \end{enumerate}
    \subsubsection{Phasenportrait}
        Die Feldlinien des Vektorfeldes $\vvec$ beschreiben die Lösungskurven des \textit{autonomen} DGL Systems.
        $$
            \vvec =
            \begin{pmatrix}
                \dot{x}\\ \dot{y}
            \end{pmatrix}
            =
            \begin{pmatrix}
                f_1(x,y)\\f_2(x,y)
            \end{pmatrix}
        $$
        \vspace{0.5em}
        $$
            y_{pp}' = \frac{v_2}{v_1}
        $$  
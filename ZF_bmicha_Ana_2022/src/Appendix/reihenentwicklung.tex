\subsection{Reihenentwicklung spezieller Funktionen}
    \vspace*{-1.7em}
    \begin{align*}
        \frac{1}{1-x} &= \sum_{n=0}^\infty x^n = 1 + x + x^2 + \dots\\
        \frac{1}{1+2x^2} &= \sum_{n=0}^\infty (-2x^2)^n = 1 - 2x^2 + 4x^4 + \dots\\
        \frac{x^2}{5-x} &= x^2 \cdot \frac{1}{5} \cdot \sum_{n=0}^\infty \left( \frac{x}{5} \right)^n = \sum_{n=0}^\infty \left( \frac{1}{5} \right)^{n+1} x^{n+2}\\
        e^x &= \sum_{n=0}^\infty \frac{x^n}{n!} = 1 + x + \frac{x^2}{2!} + \frac{x^3}{3!} + \dots\\
        \sin(x) &= \sum_{n=0}^\infty (-1)^n \cdot \frac{x^{2n+1}}{(2n+1)!} = x - \frac{x^3}{3!} + \frac{x^5}{5!} - \dots
    \end{align*}
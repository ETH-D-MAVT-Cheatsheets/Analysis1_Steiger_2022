% !TeX root = ../../ZF_bmicha_Ana.tex
\subsection{Euler DGL}
    \subsubsection{Homogen}\label{sec:Euler-homogen}
        \vspace{0.5em}
        \mathbox{
            a_n x^n y^{(n)} + \cdots + a_2 x^2 y'' + a_1 x y' + a_0 y = 0
        }
        \begin{enumerate}
            \item Ansatz \fbox{$y=x^\alpha$} einsetzen $\to$ \textbf{Indexpolynom}
            \item Nullstellen $\alpha_i$ des Indexpolynom bestimmen
            \begin{itemize}
                \item $\alpha_1 \neq \alpha_2 \neq \cdots$, reell
                    $$
                        y = C_1 x^{\alpha_1} + C_2 x^{\alpha_2} + C_3 x^{\alpha_3}+ \cdots
                    $$
                \item $\alpha_1 = \alpha_2 = \cdots$, reell
                    $$
                        y = C_1 x^{\alpha_1} + C_2 {\color{magenta} \ln(x)} x^{\alpha_2} + C_3 {\color{magenta} \ln(x)^2} x^{\alpha_3} + \cdots
                    $$
                \item $\alpha_{1,2} = \alpha_{3,4} = \cdots = {\color{red}a} \pm i {\color{blue}b}$
                    \begin{align*}
                        y =\phantom{\ln} &x^{{\color{red}a}} (C_1 \cos({\color{blue}b} \ln(x)) + C_2 \sin({\color{blue}b} \ln(x))) +\\
                            {\color{magenta} \ln(x)}&x^{{\color{red}a}} (C_3 \cos({\color{blue}b} \ln(x)) + C_4 \sin({\color{blue}b} \ln(x)))+\\
                            {\color{magenta} \ln(x)^2}&x^{{\color{red}a}} (C_5 \cos({\color{blue}b} \ln(x)) + C_6 \sin({\color{blue}b} \ln(x)))+ \cdots
                    \end{align*}
            \end{itemize}
        \end{enumerate}
    \subsubsection{Inhomogen}
        \vspace{0.5em}
        \mathbox{
            a_n x^n y^{(n)} + \cdots + a_2 x^2 y'' + a_1 x y' + a_0 y = q(x)
        }
        \begin{enumerate}
            \item Homogene Lösung $y_h$ bestimmen (\ref{sec:Euler-homogen})
            \item Partikuläre Lösung $y_p$
            \begin{itemize}
                \item Ansatz (\ref{sec:1.Ord-homogen})\\
                      Ansatz klappt nicht $\to$ mit $\ln(x)$ multiplizieren
                \item Lagrange-Methode (\ref{sec:Lagrange-2-Ordnung})
            \end{itemize}
        \end{enumerate}
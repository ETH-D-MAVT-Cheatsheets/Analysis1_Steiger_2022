% !TeX root = ../../ZF_bmicha_Ana.tex
\subsection{Koordinatentransformationen}
    \begin{center}
        \textit{kartesisch:}
    \end{center}
    \begin{minipage}{0.49\linewidth}\vspace{-1em}
        \begin{align*}
            dA = dxdy \phantom{ml}
        \end{align*}
    \end{minipage}
    \begin{minipage}{0.49\linewidth}\vspace{-1em}
        \begin{align*}
            dV = dxdydz \phantom{ll}
        \end{align*}
    \end{minipage}

    \hrule
    \begin{center}
        \textit{zylindrisch:}
    \end{center}\vspace{-0.25em}
    \begin{minipage}{0.49\linewidth}\vspace{-1em}
        \begin{align*}
            x &= r \cos(\varphi)\\
            y &= r \sin(\varphi)\\\\[0.25em]
            dA &= r \, drd\varphi
        \end{align*}
    \end{minipage}
    \begin{minipage}{0.49\linewidth}\vspace{-1em}
        \begin{align*}
            x &= r \cos(\varphi)\\
            y &= r \sin(\varphi)\\
            z &= z\\[0.25em]
            dV &= r \, drd\varphi dz
        \end{align*}
    \end{minipage}\vspace{0.5em}

    \hrule
    \begin{center}
        \textit{sphärisch:}
    \end{center}\vspace{-0.25em}
    \begin{center}
        \begin{minipage}{0.49\linewidth}\vspace{-1em}
            \begin{align*}
                x &= r \sin(\theta)\cos(\varphi)\\
                y &= r \sin(\theta)\sin(\varphi)\\
                z &= r \cos(\theta)\\[0.25em]
                dV &= r^2 \sin(\theta) \, drd\varphi d\theta
            \end{align*}
        \end{minipage}
    \end{center}